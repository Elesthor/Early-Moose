
% !TEX TS-program = pdflatex
% !TEX encoding = UTF-8 Unicode


\documentclass[11pt]{article} % use larger type; default would be 10pt

\usepackage[utf8]{inputenc} % set input encoding (not needed with XeLaTeX)

%%% PAGE DIMENSIONS
\usepackage{geometry} % to change the page dimensions
\geometry{a4paper} % or letterpaper (US) or a5paper or....
% \geometry{margin=2in} % for example, change the margins to 2 inches all round
% \geometry{landscape} % set up the page for landscape
%   read geometry.pdf for detailed page layout information

\usepackage{graphicx} % support the \includegraphics command and options

% \usepackage[parfill]{parskip} % Activate to begin paragraphs with an empty line rather than an indent

%%% PACKAGES
\usepackage{booktabs} % for much better looking tables
\usepackage{array} % for better arrays (eg matrices) in maths
%\usepackage{paralist} % very flexible & customisable lists (eg. enumerate/itemize, etc.)
\usepackage{verbatim} % adds environment for commenting out blocks of text & for better verbatim
\usepackage{subfig} % make it possible to include more than one captioned figure/table in a single float
% These packages are all incorporated in the memoir class to one degree or another...

%%% HEADERS & FOOTERS
\usepackage{fancyhdr} % This should be set AFTER setting up the page geometry
\pagestyle{fancy} % options: empty , plain , fancy
\renewcommand{\headrulewidth}{0pt} % customise the layout...
\lhead{}\chead{}\rhead{}
\lfoot{}\cfoot{\thepage}\rfoot{}

%%% SECTION TITLE APPEARANCE
%\usepackage{sectsty}
%\allsectionsfont{\sffamily\mdseries\upshape} % (See the fntguide.pdf for font help)
% (This matches ConTeXt defaults)
\usepackage{verbatim}
%%% ToC (table of contents) APPEARANCE
%\usepackage[nottoc,notlof,notlot]{tocbibind} % Put the bibliography in the ToC
%\usepackage[titles,subfigure]{tocloft} % Alter the style of the Table of Contents
%\renewcommand{\cftsecfont}{\rmfamily\mdseries\upshape}
%\renewcommand{\cftsecpagefont}{\rmfamily\mdseries\upshape} % No bold!

\title{Rapport I,  [Early Moose]}
\author{Thomas $S\pi\tau$, Olivier Marty}
%\date{} % Activate to display a given date or no date (if empty),
         % otherwise the current date is printed 

\begin{document}

\maketitle

\begin{verbatim}
////////////////////////////////////////////////////////////////////////////////
//                                                                            //
//                                                   \_\_    _/_/             //
//                                                       \__/                 //
//                                                       (oo)\_______   /     //
//                                                       (__)\       )\/      //
//                                                           ||-----||        //
//                                                           ||     ||        //
////////////////////////////////////////////////////////////////////////////////  


\end{verbatim}
\section{Structure de l'Arbre de Syntaxe Abstraite}

Les classes utilisées sont proches de la grammaire, avec en particulier~:
\begin{itemize}
 \item  \texttt{MetaProc} représente le \textbf{Proc} de la grammaire
 \item  \texttt{Process} représente le \textbf{P}
\end{itemize}

Pour \texttt{MetaProc} et \texttt{Process}, la chaîne est construite grâce au dernier élément du constructeur qui est respectivement un \texttt{MetaProc} et un \texttt{Process}. La fin est marquée respectivement par un \texttt{None} et par un \texttt{PTrivial}.
Lors du parsing, pour le processus \texttt{PIf}, le processus suivant est toujours initialisé à \texttt{PTrivial}, mais peut être remplacé ensuite si le \textbf{if} est parenthésé, comme dans le code suivant~:
\begin{center}
 \textbf{(if test then 0 else 0).0}
\end{center}

Les termes sont représentés par la classe \texttt{Term}, héritée pour chacun d'eux, sauf pour les valeurs qui héritent de la classe \texttt{Value}.

Sachant qu'il est impossible en Scala de faire hériter une \emph{case class} d'une \emph{case class}, les valeurs pourraient hériter directement de \texttt{Term} mais nous voulions pouvoir les différencier facilement, sans énumérer tous les cas. Nous avons donc opté pour une \textbf{binding class}, \texttt{TValue} qui hérite de \texttt{Term} et qui encapsule un objet \texttt{Value}.

\section{Parsing}

La grammaire simple, étant donné qu'aucune ambiguïté n'est levée tardivement, nous a laissé imaginer un parseur simple, qui construit l'AST au cours de la lecture du code.

\subsection{Structure du parseur}

La lecture du code se fait avec un objet de type \texttt{Input} qui fournit des fonctions utiles~: \texttt{peek, getWord, getNumber, checkNextWord}, etc, et qui saute les commentaires.
La méthode abstraite \texttt{getChar\_} est renseignée dans deux implémentations~: \texttt{InputFromFile} et \texttt{InputFromString}.

La classe \texttt{Input} utilise la classe \texttt{Checker} qui permet de définir facilement un alphabet légal, tout en  offrant des messages d'erreurs clairs.

La classe \texttt{Parser} utilise l'objet \texttt{Input} fourni pour créer l'AST, à l'aide de multiples fonctions dont la sémantique est immédiate (exceptée la dernière)~: 
\begin{itemize}
\item \texttt{parseMetaProc} 
\item \texttt{parseProcess}
\item \texttt{parseTerm}
\item \texttt{parseVariable}
\item \texttt{parseChannel}
\item \texttt{parseConstant}
\item \texttt{parseList}
\item \texttt{parseProcessSeq}~: cherche le '.' à la fin d'un \textbf{P}~: s'il est présent, elle appelle \texttt{parseProcess}, sinon elle renvoie \texttt{PTrivial} ce qui fait remonter la pile d'appel jusqu'au parsing d'un \texttt{MetaProc}, ou d'un \texttt{PIf}.
\end{itemize}

Toutes ces fonctions créent un objet ad-hoc, et laissent le curseur de l'\texttt{Input} juste après le dernier caractère réellement utilisé. Par exemple pour le code \textbf{in(c,x)}, le \textbf{x} engendre une variable, mais la parenthèse est laissée pour permettre à la fonction appelante, qui parse le \textbf{in}, de retrouver la parenthèse fermante. Toutes les situations similaires ont motivé la fonction \texttt{Input.peek} : regarder un caractère sans le consommer.

Le fonctionnement de \texttt{parseProcess} et \texttt{parseTerm} est basé sur les \textbf{délimiteurs}~: on peut lire un mot jusqu'à un certain délimiteur ((),=: etc) pour discerner les cas~:
\begin{itemize}
\item pour les \texttt{Process}, un \emph{pattern matching} sur le mot clé lu et le délimiteur permet de discerner facilement les cas;
\item pour les \texttt{Term}, on regarde en premier lieu s'il s'agit d'un nombre (le premier caractère est '-' ou un chiffre), sinon on effectue un \emph{pattern matching} semblable, et pour finir on recherche un éventuel opérateur en notation infixe.
\end{itemize}

\subsection{Précisions}

Trois précisions sont à faire~:
\begin{itemize}
\item les constructeurs qui demandent des valeurs attendent en réalité des termes, qui devraient, à l'interprétation, être réduits en valeurs. Aucune distinction n'est faite entre les termes réductibles en valeur et les autres, la vérification du type étant systématique lors de l'interprétation.
\item aucune constante n'est déclarée en dehors d'un \texttt{new}. Les déclaration de constantes réduisent les variables portant le même identifiant.
\item lors du parsing, les canaux sont représentés uniquement par leur nom. Ce choix permet de construire l'AST indépendamment de la stratégie d'exécution. Les détails d'implémentation des canaux sont dans la partie suivante.
\end{itemize}

Comme prévu, la parsing peut être fait au fur et à mesure~: la fonction \texttt{peek} de la classe \texttt{Input} permet de regarder uniquement un caractère sans le consommer, et c'est suffisant. La complexité du parseur est linéaire, comme on doit pouvoir l'espérer avec une telle grammaire.

\section{Interprétation du langage}

  L'interprétation repose sur une classe principale d'encapsulation~: \texttt{Interpretor}.  Celle-ci se décompose en quatre méthodes principales d'interprétation~:
\begin{itemize}
  \item \texttt{interpretValue}
  \item \texttt{interpretTerm}
  \item \texttt{interpretProcess} (interprète toute une chaîne de \textbf{P})
  \item \texttt{interpret} (procédure d'appel principale, effectuant les initialisations nécessaires (\emph{cf infra})
\end{itemize}

Remarquons que l'interprétation des processus est sémantiquement indépendante du caractère synchrone ou asynchrone~: seul le comportement des canaux diffère. Il est ainsi naturel de ne considérer qu'une classe d'interprétation, mais de distinguer le comportement des canaux comme nous le verrons ultérieurement.

\subsection{Schéma d'interprétation suivi}

\subsubsection{Paradigme de programmation}
Bien que le paradigme objet suivi lors de la conception de l'AST permettrait d'interpréter les processus, termes et autres valeurs à l'aide d'une méthode attachée à l'élément lui même, nous avons choisi de construire une \textbf{fonction d'interprétation distincte}, dans un style plus fonctionnel, en utilisant la capacité de matching associée aux case class Scala. Ce choix d'implémentation permet une modification aisé du comportement de l'interpréteur en cas de modification de la sémantique.

\subsubsection{Principe général}
L'interprétation d'un AST débute lors de l'appel de la méthode \texttt{interpret}. L'AST est alors parcouru une fois, sans détailler les termes, pour récupérer l'ensemble des canaux, créer pour chacun d'eux l'objet adéquat à la stratégie et lier le nom et le canal dans une \texttt{Map}.

Un AST commençant nécessairement par un \texttt{MetaProc}, on crée autant de threads que demandé, encapsulés dans la classe \texttt{InterpretThread} pour interpréter la chaîne de processus liée au \texttt{MetaProc}. La suite de la chaîne de \texttt{MetaProc} est ensuite évaluée récursivement, et les processus sont lancés à la volée et s'évaluent de manière parallèle.
% TODO nous vs on
\subsubsection{Environnement versus $\beta$-réduction}
Le premier choix effectué lors de la conception de l'interpréteur est le choix d'une méthode de substitution des variables. Au vue de la sémantique du langage, deux choix sont possibles~: la création d'un \textbf{environnement} qui se transmet le long de l'interprétation ou la  \textbf{$\beta$-réduction} à la volée.

Nous avons optés pour la deuxième solution, respectant au plus près la sémantique \emph{small-step} proposée.  Ce choix est techniquement possible, puisqu'il n'y a transmission de la réduction qu'aux processus situés sur la même branche que le processus courant. En outre elle est très aisée à réaliser car il n'y pas de distinctions subtiles à faire entre variables libres et liées. En outre l'implémentation de cette réduction se fait naturellement en tant que méthode attachée aux éléments de l'AST.

\subsubsection{Cas du in\^{}k}
Néanmoins une telle méthode pose quelques problèmes quant à l'interprétation d'un processus de type \texttt{in\^{}k} de manière récursive~: l'opérateur \texttt{::} ayant à la fois un rôle sémantique et syntaxique que la construction avec la classe \texttt{ListTerm} ne permet pas de représenter. Nous avons donc dépiler la structure récursive de la sémantique \emph{small-step} pour obtenir une interprétation itérative de ce processus~: la liste se construit en allant lire le canal concerné et est remplacée seulement à la fin de l'opération de lecture.

\subsection{Gestion des canaux}

La différence entre le mode synchrone et asynchrone ne s'effectuant que dans le comportement des canaux, le \emph{design pattern} \textbf{strategy} paraissait naturellement adapté à l'implémentation des canaux.

Le pattern est géré par le trait \texttt{ChannelHandler} dont dérive les trois stratégies utilisées~:
\begin{itemize}
	\item \texttt{AsynchroneStrategy}
	\item \texttt{SynchroneStrategy}
	\item \texttt{StdoutStrategy}
\end{itemize}

La première implémentation des stratégies utilisait des boucles de \texttt{Thread.sleep(1)} pour stopper les processus lorsque la sémantique l'imposait (attente d'un message si le canal est vide, attente mutuelle des processus \texttt{in} et \texttt{out} en mode synchrone), mais cette solution est hautement inélégante car consomme beaucoup de CPU inutilement, sans compter le fait qu'écrire un code robuste à la concurrence était délicat. La solution retenue est d'utiliser des \textbf{sémaphores} afin d'effectuer les blocages sus-cités.

\end{document}




